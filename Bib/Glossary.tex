\newacronym{onu}{ONU}{Organización de las Naciones Unidas}
\newacronym{pnud}{PNUD}{Programa de las Naciones Unidas para el Desarrollo}
\newacronym{ods}{ODS}{Objetivos de Desarrollo sostenible}
\newacronym{pidesc}{PIDESC}{Pacto Internacional de Derechos Económicos, Sociales y Culturales}
\newacronym{wri}{WRI}{World Resources Institute}
\newacronym{gei}{GEI}{gases de efecto invernadero}
\newacronym{nrel}{NREL}{\textit{National Renewable Energy Laboratory}}
\newacronym{wqa}{WQA}{\textit{Water Quality Association}}
\newacronym{tsd}{TSD}{total de sólidos disueltos}
\newacronym{ampp}{AMPP}{\textit{Association for Materials Protection and Performance}}
\newacronym{nasa}{NASA}{\textit{National Aeronautics and Space Administration}}

%Tecnologías de desalación
%Thermal
\newacronym{med}{MED}{destilación multi-efecto}
\newacronym{medad}{MEDAD}{Destilación multi-efecto y desalinización por adsorción}
\newacronym{msf}{MSF}{Destilación Flash Multietapa}
\newacronym{mvc}{MVC}{Destilación por compresión mecánica}
\newacronym{hdh}{HDH}{Humidificación - Dehumidificación}
\newacronym{ds}{DS}{Destilación solar}
\newacronym{frz}{Frz}{Desalinización por congelación}
%Pressure
\newacronym{ro}{RO}{Ósmosis Inversa}
\newacronym{fo}{FO}{Ósmosis Directa}
\newacronym{ed}{ED}{Electrodiálisis}
\newacronym{nf}{NF}{Nanofiltración}
%Chemical
\newacronym{iex}{I.Ex}{Desalinización por intercambio de iones}
\newacronym{lle}{LLE}{Extracción líquido-líquido}
\newacronym{ghyd}{G. Hyd}{Hidrato de gas}

\newacronym{roc}{ROC}{Radiación de onda corta}
\newacronym{pvuvc}{PV UV cut}{Poly visible (UV cut)}

\newglossaryentry{water_stress_level}
{
	name= {nivel de estrés hídrico},
	description={Extracción de agua dulce en proporción a los recursos de agua dulce disponibles; es la razón entre el total de agua dulce extraída por los principales sectores económicos y el total de recursos hídricos renovables, teniendo en cuenta las necesidades ambientales de agua. Este indicador también se conoce como intensidad de extracción de agua},
%	plural={plural name}
}

% http://www.ii.unam.mx/es-mx/AlmacenDigital/Gaceta/Gaceta-Septiembre-Octubre-2019/Paginas/sobre-la-destilacion-solar.aspx
\newglossaryentry{destilacion_solar}
{
	name= {destilación solar},
	description={Proceso heliotérmico en el cual se calienta una masa de agua, contenida en un recipiente cerrado con una cubierta transparente, por efecto de la radiación solar},
%	plural={plural name}
}

\newglossaryentry{desalinizacion}
{
	name= {desalinización},
	description={La desalinización se puede definir como cualquier proceso que elimina las sales del agua},
%	plural={plural name}
}

\newglossaryentry{luminiscencia}
{
	name= {luminiscencia},
	description={Propiedad que tienen ciertos cuerpos de emitir luz tras haber absorbido energía de otra radiación (principalmente ultravioleta) sin elevar su temperatura},
%	plural={plural name}
}

\newglossaryentry{fluorescencia}
{
	name= {fluorescencia},
	description={Tipo particular de \gls{luminiscencia} que caracteriza a las sustancias que al absorber luz a una determinada longitud de onda emiten parte de esa energía en luz a una longitud de onda más larga},
%	plural={plural name}
}